\documentclass{standalone}
% preamble: usepackage, etc.
\begin{document}

\chapter{图匹配推理引擎与符号计算推理的设计和构建}
\section{概况}

\section{复杂逻辑推理研究}

\subsection{正向推理}

\subsection{逆向推理}

\subsection{正逆结合}

\section{图匹配推理引擎的构建与设计}

\subsection{引擎基本设计思想}

\subsection{逻辑架构}

\subsection{Match Engine}

\subsubsection{图匹配}

\subsubsection{匹配原则}

\subsection{引擎的具体功能实现}

\section{类人解答过程的构建}

\subsection{类人解题的前提}

\subsection{重构类人解答}

\subsubsection{实例化定理编码}

\subsubsection{构建解答过程}



如图3-1(a)所示给出了时间步长选取为0.5ns时采用三种不同存储方式计算的平板中心处 方向的感应电流值与IDFT方法计算结果的比较。如图3-1(b)所示给出了存储方式为基权函数压缩存储方式,时间步长分别取时平板中心处 方向的感应电流计算结果,从图中可以看出不同时间步长的计算结果基本相同。

\begin{algorithm}[H]
	\KwData{this text}
	\KwResult{how to write algorithm with \LaTeX2e }
	initialization\;
	\While{not at end of this document}{
		read current\;
		\eIf{understand}{
			go to next section\;
			current section becomes this one\;
		}{
		go back to the beginning of current section\;
	}
}
\caption{How to wirte an algorithm.}
\end{algorithm}

由于时域混合场积分方程是时域电场积分方程与时域磁场积分方程的线性组合,因此时域混合场积分方程时间步进算法的阻抗矩阵特征与时域电场积分方程时间步进算法的阻抗矩阵特征相同。

\section{时域积分方程时间步进算法矩阵方程的求解}

\section{本章小结}
本章首先研究了时域积分方程时间步进算法的阻抗元素精确计算技术,分别采用DUFFY变换法与卷积积分精度计算法计算时域阻抗元素,通过算例验证了计算方法的高精度。

\end{document}
