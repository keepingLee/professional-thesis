\documentclass{standalone}
% preamble: usepackage, etc.
\begin{document}

\chapter{复杂图推理中的知识表示}
\section{概况}
本章主要是介绍图同构推理引擎在复杂图推理中的知识表示问题,主要分为
两大模块,一个是初等数学领域概念知识图谱的构建,另一方面是实例化
定理库的构建。其中概念知识图谱的构建从知识图谱构成、知识图谱存储、
知识图谱的应用三个方面介绍;实例化定理库的构建则从定理来源、定理库
创建、定理书写规范、定理的应用四个方面介绍。总体流程图如图3-1所示。
*******************
\section{概念知识图谱的构建}
\subsection{知识图谱构成}
知识图谱是用一个逻辑上的图去表征语义,这个图可以是连通的,也可以是
非连通的。本文中构建的初等数学领域的概念知识图谱就是有一些离散图
组成的逻辑上的大图。无论是离散图还是逻辑上的图,它们的最小组成单位
都是三元组结构,本文中设计的图同构推理引擎的最小推理结构就是知识
图谱中的三元组。三元组则是由两个实体和它们之间的关系构成的,实体
和关系又是三元组的最小组成单位。例如,”三角形ABC与三角形DEF相似“,
解析成三元组结构为(三角形,相似关系,三角形),这里的头尾实体均为
”三角形“,关系为”相似关系“。这里为了描述方便起见,以实体和关系的
类型字段表示,但对于后续推理来说,实体类和关系类的字段和属性信息
远丰富的多。接下来详细介绍本文如何构建实体类和关系类。

(1)实体类的构成
本文中实体都是来源于初等数学中基本概念、名词,如”函数“实体、”向量“
实体、”数列“实体、”双曲线“实体等。当前构建的实体类共236个,为了保证
实体类的可扩展性和灵活性,在最初的架构设计上采用了”开闭原则“和应用
Java语言的继承、多态以及反射的思想,即我们构建出一个抽象实体类
AbstractData,在AbstractData中添加共有的属性字段,以及相关的
函数方法,再让所有的具体实体类继承AbstractData,这样的好处有
两点:第一点可以做到很好的扩展性,可以随时添加新的实体类;第二点
方便编码,当不知道是哪个实体类时,在编译时期可以直接创建抽象父类
实体类,再利用反射技术指向不同的子类对象。

抽象实体类除了上文中描述的作用,它的一些字段在推理中也至关重要。
如图3-2是AbstractData抽象实体类的具体属性字段表。
*********************

(2)关系类的构成
关系是表示两个实体之间的关联,因此关系类中首先会记录它所关联的
头尾实体的信息,这里为了区分关系名和实体名,自然语言处理中会
统一给抽取的关系命名加上”Relation“后缀,如首项关系”FirstItem
Relation"。为了保证图推理引擎的匹配精确度,关系类构建的需要
符合以下几条准则:
1)关系的命名尽量规范,头尾实体名加上“Relation"后缀;
2)关系名的唯一性,若出现关系名重复,代码运行时会有反射异常;
3)关系构建尽量精细化,如”点在直线上“,”点在圆上“,我们应当
构建”PointOnLineRelation","PointOnCircleRelation",而不是
都构建成“OnRelation”。

同样的,与实体类建模相同的是,所有的具体关系类也都继承于同一个
抽象父类AbstractRelation。与AbstractData一样,AbstractRelation
也有关系特有的一些属性值,不同的属性值在推理中发挥着不同的用途。
下图3-3是AbstractRelation抽象实体类的具体属性字段表。
*********************
\subsection{知识图谱存储}
概念知识图谱就是由这些实体类和关系类通过转换成图谱中的节点和关系,
然后存储在Neo4j图数据库中。Neo4j是一种图数据库,图中节点可以存储
实体类相关信息,Neo4j支持的数据格式是字符串以及字符串数组,对于
每一对属性值是k-v键值对存储的形式。同样的,图中节点之间的边就是
用来存储关系类的相关信息。这里特别说明,由于Neo4j图数据库数据
存储不支持对象存储,这对后面有些属性作为对象存储并不是友好兼容,
例如,“两直线相交于点P”这是三元关系的描述,如何在知识图谱中表征
三元关系,有两种处理方式。第一种解构成一个三元组(直线,相交关系,直线)
,其中实体点P作为相交关系的属性存储;第二种是将三元关系转换成
多组两元关系,(直线,相交关系,直线),(点,在直线上关系,直线),
(点,在直线上关系,直线),这样便实现了多元关系的存储。在后续的
推理中我们大多数使用第二方式去存储多元关系。如图3-4是三元关系的
存储示意图。
***************************
\subsection{知识图谱的应用}
本文中初等数学知识图谱主要应用在自然语言理解的关系抽取中和后端
解题的实例化子图。自然语言理解是整个解题系统求解过程的第一步,
后端的推理也是依赖于前端自然语言理解的结果,因此要求自然语言
理解要尽可能地命名体实别准确以及保证这些实体之间的关系抽取的
准确度。初等数学概念知识图谱便给关系抽取提供一定的支撑,首先
命名体识别是自然语言理解的第一步,命名体识别出来的实体则对应
于知识图谱中的实体节点。自然语言理解和知识图谱是个相辅相成、
相互完善的过程:当出现自然语言理解识别出的实体在知识图谱中
没有,需要在知识图谱中补充;当自然语言理解识别的实体于
知识图谱中的不一致,需要对自然语言理解实体命名做修正。
进一步,对于关系抽取的第一步,输入任意两个实体,去概念
知识图谱中反查出这两个实体所包含的所有关系,最终抽取的
关系必须在这组关系集合中,因此知识图谱的构建对自然语言
理解至关重要。

后端解题中同样需要依赖于知识图谱。本文构建的是一个图
推理引擎,推理引擎的入口就是自然语言理解实例化出来的子图,
其中包括题目子图和规则子图。无论是题目子图和实例化子图都
必须能在概念图谱中包含其所有的实体和关系,如果在概念图谱
中找不到子图对应的实体或者关系,推理引擎则会抛出反射异常,
需要自然语言理解或者知识图谱做相应的修正和补充。

\section{实例化定理库的构建}
\subsection{定理来源}
本文要构建的是初等数学领域的实例化定理库,定理的范围是初等
数学领域,定理的来源于三个部分:第一部分是目前初中、高中的
数学教材上的定理以及公理,同时参考市面上主流教辅书籍(如
《高中数学知识清单》【】)和互联网上相关的信息;第二部分
是从标准答案的解题过程中抽取相应的解题方法、解题技巧,
为了保证自然语言理解的可靠性,标准答案会做些微调,尽量
保证标准答案中描述的实体类型明确,可以是一个短句作为一个
实例化定理,也可以是多个短句组合成一个实例化定理。第三部分
是表达式变形,表达式的恒等变形,合并同类项,拆分移项等一系列
操作转化成实例化定理。
\subsection{定理库创建}
\subsection{定理书写规范}
上文中介绍了实例化定理的来源,为了保证推理引擎的简单性和统一性,
需要对实例化定理进行规范化和标准化书写。首先,实例化定理是分成
两个部分,前提条件和结论。其中,前提条件就是原始的定理文本描述,
是实例化定理参与匹配的部分;结论部分是需要部分人工参与书写,它
是推理引擎产生知识的依据,即当该实例化匹配成功时,推理引擎会解析
定理结论部分,然后触发相应的符号计算服务,产生定理知识并依据不同
的知识更新策略,将新知识插入到原有的知识库中,进行下一轮的知识迭代。
因此推理引擎对实例化结论部分的书写提出了以下准则:

1)标准化实例化定理结论部分由三个部分组成:操作函数、定理公式、形式化
参数。操作函数是触发定理产生新知识的一组动作,操作的对象就是定理公式
以及形式参数置换后的具体参数,在书写时使用“#”将操作函数进行区分,操作
函数有时可以省略,当省略时,引擎会默认添加“simp”操作函数表示化简操作;
定义公式是操作函数操作的对象之一,它可以是书中的定理公式、等价变形或者
表达式变换,例如求等差数列前n项的通项公式“a_n=a_1+(n-1)*d","x^2+2*x*y
+y^2=(x+y)^2"等都属于定理公式的范畴;形式化参数指的是定理中抽象化参数,
不参与具体计算,需要注意的是定理公式和形式化参数之间是”:“隔开,而参数
于参数之间是”&“符号隔开。最终,完整的一条实例化定理的结论部分便书写完成。
如求等差数列通项公式的实例化定理结论为:”#simp#a_n = a_1+(n-1)d:a_1&d“

2)如图3-5列举了常见的操作函数表
**********************

3)上文中介绍过形式化参数是不直接参与计算的,主要是完成接下来形参到实参
的置换,因此实例化中的参数书写也需遵循一定的规则。

3.1)变量名的置换。所谓变量名置换,即只对具体题目参数中的名称进行替换。
如,实例化中参数为点P,题目中具体参数为点A=(1,2),变量名置换法则是
只替换A=(1,2)的变量名”A“,返回的置换结果是P=(1,2)

3.2)全称替换。有些情况下,需要获取到原始题目的所有参数信息,而不是只
进行变量名的替换,需要注意运算符两边变元数相同,全称替换在函数中应用
较为广泛。例如,实例化中参数是f(x)=g(x),题目中具体参数为h(x)=a*x^2-b,
因为运算符”=“两边的变元数相同,发生的事全称置换,置换后的结果是保留原始
题目参数h(x)=a*x^2-b;若实例化参数改写成函数f(x),则发生的置换是变量名
置换,置换结果为f(x)=a*x^2-b

3.3)变元置换。变元置换是指对表达式中的变量进行一一映射,从而产生一组
或多组轮换关系。变元置换在置换策略中应用广泛,在完成变量抽取后,对变量
进行轮换映射参与后续的计算模块。本文中的变元置换具体表现为:数列角标的
置换、函数的自变量置换、系数置换等。例如,f(x)=a*x^2+b*x+c和
f(x)=x^2+9*x-3所对应的一组轮换映射为{a=1,b=9,c=-3},a_m+a_n=p和
a_3+a_5=0所对应的一组轮换映射为{m=3,n=5,p=0}

\subsection{定理的应用}
在整个问题求解系统中,解题是最核心的部分,是系统的功能需求和第
一步目标,而解题的核心部分是推理引擎,图推理引擎设计的驱动依据就是
实例化定理。定理驱动方式分为两种,一种是针对所有实例化定理采用循环
驱动,这是一种外部驱动;第二种针对单个实例化,是内部驱动方式。图推
理引擎会对实例化定理子图进行解构,将实例化子图拆分成逻辑上独立的两
个子图:条件子图和结论子图。其中,条件子图作为引擎匹配的依据,结论
子图作为引擎产生新知识以及如何插入到原知识库的依据。当匹配成功时,
才会触发引擎对结论子图的处理,否则会进入到外部循环驱动,进入到
下一条规则的匹配。因此定理是引擎解题的重要依据,包括如何在定理库
中搜寻一条或多条解题路径。

实例化定理除了在解题中应用,同样应用在知识点标注上。在自然语言
处理过程中,对实例化定理添加了两个字段”1“,”2“。”1“代表该实例
化定理涉及到的相关知识点,知识点来源于初高中教材的大纲知识点,
由于一条定理可能包含多个知识点,因此字段”1“设计成List<String>
数据结构。”2“表示实例化的定理名称,主要是用于后续解答过程的答案
输出,同时实例化标签label加上定理名称便可形成对定理的唯一标识。
解题是知识点标注和类人过程输出的基石和前提,它们之间的关系具体
表现为:推理引擎在对实例化定理库进行搜寻的过程中构建解题的
定理树,对定理树进行回溯查询得到类人解答过程,定理树是记录
解题过程中所使用的实例化定理之间的逻辑关系,通过”1“、”2“
字段输出本题涉及到的所有知识点内容。


本文要构建的是初等数学领域的概念知识图谱,知识的范围是初等数学领域,
知识的来源主要是目前初中、高中的数学科目的教材。同时参考是市面上
主流教辅书籍(如《高中数学知识清单》【】)和互联网上相关的信息。
接着是实体和关系的抽取,因为该知识图谱主要是为了给后面的自动求解
系统作为基础的,而数学的解题需要又准确的信息,自动或者半自动生成
知识图谱的方法【】达不到我们需求的准确度,因此采用的是人工的方式
构建其中的实体和关系。

知识图谱最基本的组成单元为三元组,而每一个三元组是由两个实体和一个
关系构成,因此现将获取到的知识经过自然语言理解抽象成三元组结构,再
来提取出实体和关系。例如对于描述”两圆内切,直线和圆相切“可以抽取成
三元组(圆,内切,圆)和(圆,相切,直线),接着将每一个三元组结构
拆成两个实体和一个关系,例子中第一个三元组包含两个实体都是”圆“,关
系为”内切“,第二个三元组中包含的两个实体分别是”圆“和”直线“,关系
为”相切“。

在将这些文本表述抽象成三元组的过程,不能够是具体的实例,例如”直线
AB和直线CD垂直“,抽象出来的三元组为(直线,垂直,直线),而不能
是(AB,垂直,CD)。三元组中的关系抽取原则是精细化,这对后面的解题
是非常有帮组的,例如”三角形相似“和”相似“这两个关系,为了保证关系
抽取的精确性,将关系”三角形相似“从”相似“关系中单独列举出来,便于
后续的推理。

实体和关系的界限有时候不是那么分明,有的自然语言中描述的知识同时
是实体和关系,例如”线段AB是三角形ABC的边“这句表述抽象出来的三元组
为(线段,边,三角形)。这这个三元组中”边“是关系,但是在”边AB的
长度为a“这句话中,抽象的三元组为(边,长度,表达式),在这里面”边“
是作为实体的。这两个三元组放在同一个图中时,处理方法就是让”边“作为
实体和关系同时存在,并且会通过第二个三元组为”边“关系添加一个属性,
”边的长度“是该属性的描述,对应的类型为”EXpress“,即实体”表达式“的
英文名。保留”边“的实体是为了是表达变得更加的简单明了,可以降低后续
推理的复杂度。
\section{知识表示}。
知识表示是用实体类和关系类来分别表示实体和关系,每一个实体对应一个
实体类,每一类关系对应一个关系类。
\subsection{实体类的结构}
(1)创建和命名
作为实体的主要是初等数学中的概念、名词,例如正方形,三角形等,类的
名字就是该实体的英文名字,命名时要保证每个名字的唯一性,命名的规则
时该实体对应的英文单词且首字母要大写,如果出现多个单词的组合,则每
个单词的首字母都要大写,例如实体”正方形“类的名字为”Square",实体
“奇函数”的类名为“OddFunction”。实体中还包含有实体的中文名,在类
中以一个静态方法所获得,中文名就是改实体的中文含义。在创建实体类
的时候,还会对实体类进行分类,相同类的实体会放在同一个包内,所有
的实体都会放在"实体“包下,在这个包下还会按照实体不同的类型进行
更细的分类,例如实体”一次函数“和实体”二次函数“都会建在”函数“这个
包下面,对于使用频率比较高的类型单独建立一个包,例如出现非常频繁
的三角形和四边形,会将它们从多边形包中单独提取出来,即”三角形“、
”四边形“、”多边形“包是平级的,而不是概念上的三角形、四边形属于
多边形,这些包名也要保证唯一性。
(2)属性表

\subsection{关系类的结构}
(1)中英文的命名方式
关系所表示的是两个实体之间的联系,每一种关系都是一个java类,和
实体一样,类的名字也是关系名,在命名规则上也是类的名字对应的是
该关系的英文名字且英文单词的第一个字母大写,如有多个单词组成,
也是每个单词的首字母大写,和实体的类名不一样的是在英文单词的
后满加上”Relation"作为后缀,这样做的目的是为了在图谱生成模块
让程序快速区分实体和关系。例如关系“三角形相似”的类名为“Triangle
SimilaryRelation",关系的类中也有中文名,其在类中的字段和命名
与实体的规则是相同的。
(2)开始和结束实体表
(3)属性表
关系可以很容易的表示两个实体之间的联系,但是在许多的表达和描述中
不仅仅是两个实体之间的关系,还有可能是三个实体甚至更多的实体之间
的关系,为了解决这一问题,为关系添加了属性,并且图数据库Neo4j也是
支持关系中添加属性的。属性分为两个部分,其中一部分是该属性所代表的
含义,这个含义的描述是不能相同的,另一部分是该属性的类型,这里的类
型指的是实体或者关系,即类型必须为已经存在的某一个实体或者已经存在
的某一种关系。例如对于这样的描述”两条直线相交于一点“,这里的关系
是”相交关系“,开始实体和结束实体都为实体”直线“,相交的这个”点“就是
放在”相交关系“的属性中,即这个属性的描述为”相交的点“,类型为”点“
或者”Point“,在构建是统一使用的是英文名。属性表的结构示意图如图
3-3所示。
**************************
(4)特殊关系

\section{知识存储}

\subsection{生成知识图谱}

\subsection{知识图谱维护}}

\section{知识可视化}

\section{概念知识图谱的应用}



如图3-1(a)所示给出了时间步长选取为0.5ns时采用三种不同存储方式计算的平板中心处 方向的感应电流值与IDFT方法计算结果的比较。如图3-1(b)所示给出了存储方式为基权函数压缩存储方式,时间步长分别取时平板中心处 方向的感应电流计算结果,从图中可以看出不同时间步长的计算结果基本相同。

\begin{algorithm}[H]
	\KwData{this text}
	\KwResult{how to write algorithm with \LaTeX2e }
	initialization\;
	\While{not at end of this document}{
		read current\;
		\eIf{understand}{
			go to next section\;
			current section becomes this one\;
		}{
		go back to the beginning of current section\;
	}
}
\caption{How to wirte an algorithm.}
\end{algorithm}

由于时域混合场积分方程是时域电场积分方程与时域磁场积分方程的线性组合,因此时域混合场积分方程时间步进算法的阻抗矩阵特征与时域电场积分方程时间步进算法的阻抗矩阵特征相同。

\section{时域积分方程时间步进算法矩阵方程的求解}

\section{本章小结}
本章首先研究了时域积分方程时间步进算法的阻抗元素精确计算技术,分别采用DUFFY变换法与卷积积分精度计算法计算时域阻抗元素,通过算例验证了计算方法的高精度。

\end{document}
