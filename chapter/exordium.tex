\documentclass{standalone}
% preamble: usepackage, etc.
\begin{document}

\thesischapterexordium

\section{研究工作的背景与意义}
人工智能(Artificial Intelligence,AI)作为研究、开发用于模拟和扩展人类智能的理论、方法以及技术
和应用系统的一门新兴科学\citing{sun2013ren}。人工智能的概念1956年的研讨会正式被提出开始,它的发
展可谓是起起落落充满曲折坎坷。时至今日,人工智能在科研、教育、医疗、金融等各方面正大放异彩。无论
是理论还是应用层面都取得了不错的进展。

%计算电磁学方法\citing{wang1999sanwei, liuxf2006, zhu1973wulixue, chen2001hao, gu2012lao, feng997he}从时、频域角度划分可以分为频域方法与时域方法两大类。频域方法的研究开展较早,目前应用广泛的包括:矩量法(MOM)\citing{xiao2012yi,zhong1994zhong}及其快速算法多层快速多极子(MLFMA)\citing{clerc2010discrete}方法、有限元(FEM)\citing{wang1999sanwei,zhu1973wulixue}方法、自适应积分(AIM)\citing{gu2012lao}方法等,这些方法是目前计算电磁学商用软件
%\footnote{脚注序号“\ding{172},……,\ding{180}”的字体是“正文”,不是“上标”,序号与脚注内容文字之间空1个半角字符,脚注的段落格式为:单倍行距,段前空0磅,段后空0磅,悬挂缩进1.5字符;中文用宋体,字号为小五号,英文和数字用Times New Roman字体,字号为9磅;中英文混排时,所有标点符号(例如逗号“,”、括号“()”等)一律使用中文输入状态下的标点符号,但小数点采用英文状态下的样式“.”。}
%(例如:FEKO、Ansys 等)的核心算法。由文献\citing{feng997he, xiao2012yi, clerc2010discrete}可知

\section{时域积分方程方法的国内外研究历史与现状}
时域积分方程方法的研究始于上世纪60 年代,C.L.Bennet 等学者针对导体目
标的瞬态电磁散射问题提出了求解时域积分方程的时间步进(marching-on in-time,
MOT)算法。

\section{本文的主要贡献与创新}
本论文以时域积分方程时间步进算法的数值实现技术、后时稳定性问题以及两层平面波加速算法为重点研究内容,主要创新点与贡献如下:

\section{本论文的组织结构}
本文将用6个章节介绍论文的研究成果,具体论文的章节结构安排如下:

第一章:绪论。这部分主要是先描述了整个人工智能的发展状况和背景,再
描述了逻辑推理在历史上的一些发展状态,同时结合国内“互联网+教育”的
发展机遇对初等数学自动求解系统的意义进行了简要介绍。

第二章:相关理论技术。本章对文中设计的基本理论与技术进行详细的介绍
与讲解。首先,介绍了知识图谱表示知识的基本原理。然后从产生式系统原
理入手,针对论文工程期间使用的开源规则引擎Drools进行了详细的介绍。
然后介绍符号计算引擎的相应情况。这些基础理论知识是论文研究的基石,
同时也为系统设计实现提供了重要的理论保障。

第三章:图匹配推理引擎与符号计算推理的设计和构建。本章主要研究复杂
图匹配推理引擎与符号计算推理的整个逻辑架构与核心算法设计,针对出现
问题给出相应解决方案。主要从图匹配推理引擎的研究与构建,图匹配推理
引擎与计算推理的交互推理,类人解答过程的生成三个方面着手。其中以图
匹配推理引擎的构建为重点。讲述了三种不同的复杂逻辑推理组织方式,并
采用“正逆结合”的推理方式构建推理引擎,随后研究了推理引擎与符号计算的
交互推理模式,符号计算平台提供的计算服务为具体的初等数学问题的计算
打下了支撑。类人解答过程的生成,在推理的基础上,设计的DFS的搜索
回溯算法,重构类人解答过程。

第四章:图匹配推理引擎与符号计算推理在初等数学中的应用。本章主要
介绍将图匹配推理引擎运用到具体的初等数学问题求解中。首先概述基于
此推理引擎设计的类人求解系统的各个模块,然后分模块详细介绍了实现
细节。

第五章:测试与分析。对基于本论文实现的图匹配推理引擎与符号计算引擎
以及类人求解系统进行了详细测试和分析,首先测试了推理引擎在不同的
题型中求解成功率。再采取自动解答+答案标注抽取两批共200道题,然后
对结果进行了详细统计。最后对全量测试数据进行了综合统计,并根据分析
结果,指出系统中存在的一些不足,以便在今后的工作中进一步提升推理
引擎解题的准确率。

第六章:总结和展望。首先对全文的研究工作做出了总结,指出研究的主要
成果和创新点。再对研究中存在的不足做出说明,指出了有可能取得突破的
一些方案。


\end{document}