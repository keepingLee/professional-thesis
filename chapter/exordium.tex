\documentclass{standalone}
% preamble: usepackage, etc.
\begin{document}

\thesischapterexordium

\section{研究工作的背景与意义}
自1956年人工智能概念提出以来,全世界的专家学者都一直致力于人工智能不同领域的研究\citing{li2018ren}。
经过长达半个多世纪的发展,它的发展历程可谓是几经波折,经过短暂辉煌,也同样出现
过很多质疑和冷落。从开始的被质疑到后期被理论证明再到最后的广泛应用,
人工智能经历了三个不同阶段,最起初属于机器智能时代,早期最为新兴学科
吸引越来越多的科研人员参与研究,很快相继出现了一批显著的成果,如
机器化定理证明\citing{wu1997chu}、跳棋程序、求解程序等,但由于机器翻译等应用的失败,
人工智能进入了发展的停滞期。后面神经网络飞速发展,人工智能进入到感知智能时代,
各种不同的神经网络模型加上搭载一系列的传感器应用于不同领域,人脸识别、多任务分类、图像处理等领域取得成果
有目共睹。随后,人工智能进入到第三阶段,认知智能。所谓认知智能,即机器具备像人类一样
具有独立思考、理解、以及逻辑推理等认知能力,机器学会逻辑推理是人工智能发展的终极目标。
认知智能的潜在应用领域非常广泛,无人驾驶、智慧医疗、金融风控、机器老师等领域都急需
认知智能取得突破。

近些年来,人工智能捷报频传。在金融风控和交易领域,
利用其强大的计算能力和存储能力,在银行股市等行业它将取代人们成为新的交易员。
相比于人类交易员,计算机不仅能出色完成数据管理和计算,还能依据大数据挖掘技术
进行风险预测和价值评估。根据美国花旗银行的最新研究报告,从2012年截至2014年底人工智能投资管理
顾问的资产增值量接近140亿美元\citing{song2016ji}。
在后面的几年时间内这个投资管理的资产额度以指数级别增长,预计总额度将达到5万亿
美元。在竞技体育方面,人工智能同样大放异彩。Edge Up Sports 与 IBM 团队合作,前者
提供了球员的上场时间、进球数、每场触球平均时长,以及比赛场地、时间、天气等外在客观因素
等所有能影响比赛的数据,IBM则利用这些大数据使用相关模型进行大数据分析和预测比赛趋势。
同样通过大数据挖掘可以为教练员提供更直观的每个球员每场的表现,从而为每个球员制定不同的训练
计划表。无人驾驶领域更是当前热门的研究领域,各大公司和科研都想在无人驾驶方面能取得突破性
成果。无论是谷歌的无人驾驶汽车已经完成了在美国各种不同城市不同路况下测试,虽然也会发生一些
判断失准导致的一些事故问题,还是国内的百度人工智能研究院的智能汽车,小马智行等公司相继投入
无人驾驶的研发中。人工智能还有很多成熟的应用场景,比如个人助手,苹果的sari,微软的cortana
都是人工智能在语音的应用。

人工智能初期阶段的应用之一就是机械化定理证明,随着人工智能不断的发展并在各个领域取得
显著的成绩时,却发现机械化定理证明多年没有实质性进展,人工智能在认知推理方面表现的
不如人意。事实上,几何机械化问题最早可以追溯到17世纪著名大数学家、大思想家莱布尼茨
就已经提出机械化证明定理的想法,但莱布尼茨并没有将这种设想形成数学形式。直到19世纪
大数学家希尔伯特及其同时代的其他数学家完成数理逻辑的创立和发展,从而完成了几何定理
数学化的形式。随着20世纪中期计算机的出现和发展,才让定理机械化证明有了实际可以实现
的可能。从最开始希尔伯特的著作《几何基础》中提出“对于可机械化证明的定理H和P,需要
满足的特征是部分的代数关系是对于局部变量必须线性”,再到20世纪30年代,美国数学家
J.F.Ritt提出了代数几何的构造性理论\citing{yan2004Grobner},这一理论对后面很多数学工作者具有启发式意义,
其中包括我国著名数学家吴文俊和他的“吴方法”也受他的理论影响。“吴方法”是20世纪70年代末
吴文俊先生在中国传统机械化思想的影响下,并研究了Ritt等人的理论工作,针对几何定理机器证明问题
给出了新的方法\citing{liao2009xu},该方法具有高效性和实用性,能够切实应用计算机上进行程序式验证。通过
大量的机器证明,吴方法在国际上引起了很大的轰动,从而掀起了机器证明新的高潮。


近些年,随着教育资源的分配不公化、不均匀,教育成本过高等一些列问题的暴露,如何
将人工智能应用于教育培训领域成为新的热门话题。人工智能赋能于教育是人们开始
追寻和探索的方向。“人工智能+教育“具体有五个方面的应用,第一阶段是自动搜题,
可以帮组学生从图库中搜索原题的解答;第二阶段是相似题推荐,主要是依赖于深度
学习模型和相似度比对算法,在题目中寻找相似题,可以让学生加强巩固同类型题目;
第三阶段知识点推荐,主要依赖于推荐算法,对学生和老师进行精准的知识点推荐,
老师减轻备课的压力,学生可以有重点的学习;第四阶段:自动判卷,辅助老师去
批改学生作业并生成作业批改报告;第五阶段:机器老师。机器老师的核心是自动
求解题目,而自动求解的核心便是自动推理,是认知智能的范畴,然后融合前四个
阶段的应用,可以是一个于学生可交互的应用于解题、讲题、推荐于一身的机器人。
前三个阶段目前均有不同程度的应用,第四、五阶段目前还是处于研究实验状态。
本文的研究内容重点就是第五阶段自动解题系统的构建和设计,希望此推理系统
可以在人工智能应用于教育领域发挥一定的作用。



\section{研究现状}
\subsection{知识图谱研究现状}
在新一代人工智能发展阶段,即认知智能阶段。由于专家系统的出现,让人们重新认识到
知识的重要性。人类的语义包含丰富的信息,这些语义信息可以转换成知识进行传递和交流。
而人工智能的大数据挖掘还是处于低纬度的特征空间,这就导致机器学习与人类语义之间
存在鸿沟。如何让机器去理解人类语义,最重要的是解决知识表示问题,而计算机知识表示需要载体,
知识图谱的出现很好起到语义信息到知识表示的桥梁。知识图谱可以承载非常庞大的语义系统,存储
是以图的形式来表示知识,图的基本构成是点(Point)和边(Edge),而点表示的是一种概念”实体“,
例如一些名词概念”人“、”动物“、“天空”、“海洋”等,边表示的是这些概念实体之间的关联,即
“关系”,例如概念实体“老虎”与概念实体“动物”之间的关系为“属于关系”,这样通过关系来描述
所有实体之间的关联,知识图谱是一张描述不同概念实体之间关联巨大网络图。知识图谱的优势在于
将数据的粒度从文本形式降到data形式,有利于知识的形式化表示和快速的搜索查询。因此可以基于
知识图谱做知识分类、知识聚合、知识推理等一系列任务。

知识图谱从最早谷歌提出概念到今天近10年的时间,它的发展和理论研究都是一路高歌猛进,
同时知识图谱应用也非常广泛,已经渗透到各行各业以及不同的学科领域。最早应用于语义搜索领域,
语义搜索是被称为互联网之父的Tim Berners-Lee 在2001 年《科学美国人》
(Scientific American)上发表的一篇文章中首次提出的概念\citing{jin20080XML,guge2015yu}。在这篇文章他解释了语义搜索的本质含义。“
语义搜索的本质是不在是搜索引擎输入框中输入的表层含义,它是可以进行深层搜索的一种
深层知识关联度的查询,其中包括联想搜索和同义词语义替换”。
传统的语义搜索存在搜索响应慢、无法进行同义词搜索、无法深层搜索等问题,这些问题的根本
原因在于知识表示不够完善。因此谷歌才提出知识图谱的概念,试图使用知识图谱来表征语义
知识,将搜索引擎的算法迁移到对图上语义的搜索,图表示语义更加准确、信息化更加丰富。
同时图的搜索效率相比于之前的文本搜索会提高不少,可以很好的优化搜索效率,提高响应速度。
知识图谱的搜索过程表现为:通过理解用户的输入描述,抽取描述中的实体和关系,在去匹配
知识图谱中的实体和关系,寻找一定的关联度信息,将查询的结果再重新返回给用户。
在问答系统(QA)领域,所谓智能问答是指通过问答式对话,让机器具备像人类一样常识
知识的理解并完成交互。不同的智能问答系统,它具有不同的功能特点,依赖的知识图谱库也
有所不同。例如聊天机器人的知识图谱库就是来源各种常识、百科等构建起来的,机器老师的
知识图谱库就来源于数学概念、定理、公理等。这是行业知识图谱,可以为行业内人提供
专业快速的知识查询响应。问答系统时语义搜索的一种延伸,在具备知识图谱的搜索功能推理,
还需要考虑上下文的语境,和问答的逻辑性。在个性化推荐领域,根据用户的浏览和消费记录
产生的数据构建商品和消费者以及商家之间的关联,生成知识图谱库,再利用基于知识图谱的
推荐算法为用户推荐感兴趣的产品或者内容。

\subsection{推理引擎研究现状}
在任何的推理系统中,推理引擎都是系统完成推理功能和驱动的核心组成部分。推理引擎
又称推理机,通常情况下推理机由调度器,执行器和一致协调器三部分组成\citing{wang2011jiyu,zhang2004jishu}。调度器是
负责控制引擎的整体流程走向,不同逻辑层和各个模块之间的连通性,相当于推理引擎的
大脑,可以起到决策和判断作用。执行器可以视为一组动作集,动作集包括
原始事实库的读取和插入,分支选取策略的执行,循环遍历和搜索的执行。一致协调器是
为了保证生成的知识保持前后一致,不会出现丢失、改写、被覆盖等情形。规则推理引擎是基于产生式系统
思想构建的,由于其逻辑结构清晰,规则独立,编程友好等优点逐渐成为主流设计框架。其核心思想在于将匹配规则单独抽离出来形成规则脚本或者存入到数据库中,
使得规则的修改、添加、删除都不需要改动业务逻辑本身,引擎会对已知事实与
规则库进行匹配搜索,产生的新知识继续添加到已知事实。

从上世纪70年代初开始,斯坦福大学使用LISP语言研发了世界上第一个
规则引擎-MYCIN 系统\citing{zhangrun2016ji},这个系统是最早的规则引擎的雏形,它首次提出
将规则知识从系统中抽取出来,通过热插拔的方式去加载规则,提高了系统的
灵活性和开放性。该系统在实际应用中也取得了不错的效果。20世纪90年代,
随着面向对象技术的兴起,为了解决大型应用软件系统的复杂度问题,开始
引入对象封装思想,分类思想,以及跨平台通信机制,同时也为规则的独立性
和外部程序封装,以及跨平台交互提供了实现的技术支撑。面向对象的编程
思想很好解决规则作为一个个的独立对象存在,可以将规则的所有信息
封装成对象,然后独立于业务流程之外,方便规则的扩展。
 21 世纪后,规则管理技术日趋成熟,2000 年 11 月,
 JavaCommunity Process组织开始着手起草 Java 规则引擎的API标准,即JSR94 规范\citing{ma2009Drools,
 xu2014SCADA}。
 在JSR94完成正式定稿的同时,市场上相继推出支持JSR94规范的规则引擎\citing{gu2013yewu},有商用和社区开源引擎,
 较为成熟的商用引擎代表ILOG Jrules和开源引擎代表Drools,它们的共同点在于引擎稳定高效,规则
 匹配速度快,从而迅速得到广泛应用。


 时至今日,规则引擎也面临一些挑战和问题。第一方面,规则标准化配置问题。
 大部分的规则库都是后端相应开发人员构建的形如xml格式的配置文件,这些配置
 文件需要较高的专业水平才能书写,行业外的人很难参与规则库构建,如何
 让规则库构建更见简易化和方便,让不具备专业知识的人也能参与构建是需要解决的
 问题。第二方面,规则的维护。需要给每个规则设置唯一的标签,通过唯一的标签
 快速定位到需要修改、删除、完善的规则。同时需要保持规则之间的独立性,防止
 删除一个规则导致别的规则无法使用。第三方面,引擎的推理效率。这里效率问题
 主要是时间开销,当规则库中规则越来越多的时候,引擎对规则库采用盲目式搜索
 的时间成本会提升,因此对规则库进行分类和适当减枝是非常有必要的,而减枝策略
 选取成为研究的重点。

\section{研究内容}
本论文主要研究基于初等数学的类人解题系统中推理引擎的设计和实现,通过对
主流的一些开源引擎调研,对比分析不同推理引擎的优缺点和它们的设计思想,
再结合本文中独有的知识图谱表示法,最终设计和构建了一个同时融合逻辑推理
和计算推理的图推理引擎。

对于完整的类人解答系统来说,必须包含四个部分:自然语言理解,知识
表示,自动推理和求解,类人解答过程输出。自然语言理解是处理人类的
文本描述中所包含的语义,将这种语义理解成计算机可以处理的形式,因此
这个部分是这个求解系统的最基础部分,后面的知识表示和自动推理都依赖
于自然语言理解的结果。对于知识表示是系统关键问题,它是衔接自然语言理解
和自动推理的桥梁,同时知识表示的合理性和简洁性对后续的推理至关重要。
表示的过于简单会无法正确表示语义,会出现信息丢失的情况;表示的过于
复杂,一方面会导致知识冗余现象,同时会急剧增加推理的复杂度。因此,
最终选用知识图谱来表示知识,知识图谱表示知识相对于传统的谓词逻辑的最大优点在于
图结构可以表示很复杂的语义结构,并且寻找任何两个实体之间的关联
可以转化成图路径算法,同时知识图谱便于展示知识特征,可以很清晰
的看出所有的实体和关系。

自动推理和求解部分最核心是图推理引擎的设计。图推理引擎是针对于
知识图谱的独立自主开发的推理引擎。从推理方式来说,它融合有
逻辑推理和计算推理;本身的逻辑架构来说,引擎包含四层设计结构:
图结构层、图匹配层、参数置换层、知识更新层;从设计思想来说,
引擎中采用是产生式系统的模式,即对于一组已知事实集合,将已知
事实与规则库中的知识进行不断匹配,从而产生新知识,在将产生的
新知识重新插入到原始的事实集合中,进而对已知事实集合进行新
的迭代,直至到达求解目标终止迭代。

综上所述,本文研究的内容具体可以概括成以下五个部分。

(1)图推理引擎中所使用的逻辑推理方式

常见的逻辑推理方式分为三种:正向推理、逆向推理、双向推理\citing{li2007zhishi,hanyi0mian}。其中,正向
推理是,从已知事实出发,通过匹配规则库产生新知识,然后不断迭代的过程,
是一种盲目式搜索的推理方式;而逆向推理恰好相反,它是从目标结论
出发,是不断规约的过程,从结论一直往前搜索已有的规则,直到满足所有
条件停止规约,是搜索解题路径的一个过程;双向推理则是融合正向推理
和逆向推理的混合推理模式,主要包含三种形式:“先正后逆”、“先逆后正”、
“正逆同时”。本文的图推理引擎采用是“先逆后正”的方式,即先对已知事实
进行逆向搜索去寻找解题路径,然后再使用正向推理对解题路径中所涉及到的
规则进行遍历,从盲目是匹配转变成精确匹配规则库中特定规则,很大程度
提升了匹配效率。

(2)逻辑推理与计算推理的交互

因为本文研究的是初等数学问题,必不可少其中涉及很多的数学公式定理的
演算,这部分需要符号计算平台提供计算服务。当对定理公式计算时,同样
会产生新知识,如何将计算新知识与逻辑推理产生的新知识进行交互,是图推理
引擎的一个重要研究内容。最终引擎设计的模式是二者产生的知识相处融合,
相互迭代,让推理的方式更加的灵活,也增强了推理引擎的求解能力。

(3)图推理引擎的逻辑结构和核心算法设计

图推理引擎最大的创新点和亮点在于该引擎是针对于图上的知识表示而非谓词逻辑
并且将传统的逻辑推理应用于图上。因此传统基于谓词表示法的推理引擎中的
匹配模式就会失效。本文构建的图推理引擎最核心的算法思想是图匹配,即如何
判断子图匹配的算法。逻辑架构也是基于图匹配的核心思想搭建的四层结构,
每一层之间既能保持相对独立,同时又存在一定的依赖和交互。虽然采用了
分层的结构,但在运行的过程每层之间是互通的,它们存在数据流的传递
和交互,数据流可以跟踪到数据流向每层前后的变化,方便定位到是哪个
逻辑层可能存在潜在bug。

(4)实例化定理库的构建

首先实例化定理库是定理实例化成的一个个的定理知识图谱,定理库是
引擎的主要驱动依据,也是引擎匹配算法的对象之一,产生新知识和插入知识的
来源之一。为了保证引擎的简单统一,实例化定理的书写需要满足一定的
规范标准,在规范标准之内,引擎增加了一些预处理机制,增加了书写的
灵活性。除此之外,在构建定理库时对实例化定理进行分类和标注,便于
后续可以高效的匹配某一类实例化即可,降低引擎匹配的时间开销。

(5)重构类人解答过程

在保证已经正确求解出答案和系统正确停机的基础上,引擎需要对推理
的过程进行重现形成解答过程,这点类似人类的答题。需要将相关的推导
过程展现出来,对于引擎就是将推理中使用的实例化定理以及计算中
涉及的化简、解方程等技巧,用“因为***,所以***”的形式呈现。
生成可读过程的方式有两种:后台日志的方式和知识图谱。解答过程
的知识图谱需要注意的是,通过添加因果关系来表示前后逻辑。
\section{本论文的组织结构}
本论文共分为6个章节来详细介绍研究成果,具体论文的章节结构安排如下:

第一章:绪论。本章分成三个部分介绍初等数学类人解答系统的发展现状和背景,
第一部分主要是介绍人工智能的发展历程以及不同发展阶段取得的重大研究成果,另外给出当前人工智能面临的一些难题;
第二部分描述了自动推理的发展历程以及不同阶段的技术手段和算法思想;第三部分是对当前
的“人工智能+教育”的现状进行研究和分析。

第二章:相关理论技术。首先,第一部分描述的是知识表示的几种不同方式以及
知识图谱表示知识的特点。第二部分是详细介绍了产生式系统,包括基本原理,组成和实现,
本文的推理引擎的设计就是基于产生式系统。第三部分介绍了推理引擎的相关算法以及
图匹配相关算法。最后介绍了推理引擎中使用的符号计算服务平台Maple。这些基础理论和技术
是推理引擎设计思想的基石,也为推理引擎的设计实现提供了重要的技术支持。

第三章:复杂图推理中的知识表示。本章主要研究在复杂图推理中的知识表示,
知识存储等问题。首先研究的问题是知识来源,这是知识表示的第一个环节,
本文的知识分为两种知识:知识点知识
和实例化定理知识。其中知识点知识来源于初高中教材中的数学名词、概念、以及一些
新定义;实例化定理来源初高中教材的公理、公式、以及标准答案的解题解题技巧和
公式变形。然后研究的是对于纯文本描述如何进行语义理解,转换成机器可以
识别的语义知识,自然语言理解。利用自然语言
理解的相关技术进行命名体识别和关系抽取。自然语言理解之后需要进行知识表征,
本文知识表示选取的是知识图谱。第二部分介绍了实例化定理库的构建,以及
实例化定理在推理引擎中的书写规范。最后详细描述实例化定理的置换原则。

第四章:图同构的数学推理引擎的设计和构建。本章主要研究基于图同构
的图推理引擎逻辑架构和图匹配算法的设计,以及推理引擎如何将复杂逻辑推理
与符号计算推理相融合。
推理引擎主要从三个方面详细介绍,第一层面是从逻辑推理方式,描述了几种不同逻辑
推理的算法思想和引擎中”先逆后正“的推理方式;第二层面从推理引擎
的基本原理和整体逻辑架构,宏观上讲述引擎的整体设计;第三层面是
推理引擎每一逻辑层的具体模块实现,以及一些重要模块对应的算法设计;
最后,对推理引擎的输出重构
类人解答过程和生成结果过程知识图谱。


第五章:测试与分析。本章的测试分为两个模块:单例测试和批量测试。
单例测试的主要内容包括两个部分:连通性测试和类人解答过程测试。
连通性测试主要是检验前端自然语言模块和后端自动推理连通性,推理引擎
不同逻辑层间的连通性,推理引擎与外部接口之间的连通性;类人解答过程 
测试包括能否输出所有涉及的定理和公式,输出的定理之间的逻辑关系是否
正确。批量测试主要是检测推理引擎每个模块和核心匹配算法的稳定性,
以及对于不同题型的兼容性。并对批量测试的结果进行了详细的统计和分析,
从解题准确率、不同题型的兼容率、每题的平均解题时长等几个不同维度
来评判推理引擎的性能和发现可能存在的问题,以便更好的提升推理引擎的
解题能力和稳定性。

第六章:总结和展望。首先对本文中的研究内容和成果做出了总结,
并结合测试结果和测试数据进行不同角度的考察和分析,找出了推理引擎
存在的不足,依据不足提出相应的解决方案。对于一些疑难问题也给出
了自己的思考。


\end{document}