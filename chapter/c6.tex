\documentclass{standalone}
% preamble: usepackage, etc.
\begin{document}
	
\chapter{总结与展望}
\section{全文总结}
本文主要工作是设计和实现了基于一个图同构的初等数学的推理引擎。工作内容主要
分成三个部分:知识图谱构建、实例化定理库构建以及图推理引擎的逻辑结构设计和模块化构建。
知识图谱是知识表示的载体,因此构建一个合理完备的初等数学知识图谱是自然语言理解和自动推理
的关键一步。知识图谱的构建分为实体构建和关系构建,然后将构建的实体类和关系类通过程序化
方式生成对应的图节点和边,存储在neo4j图数据库中。为了实现添加实体类和关系类的可扩展性
和开闭原则,构建了抽象实体类和抽象关系类,所有的实体类都继承于抽象实体类,所有的关系类
都继承于抽象关系类。第二部分是实例化定理库的构建,图推理引擎在推理方式上还是规则引擎
的一种,因此规则库是引擎驱动和产生新知识的重要依据。实例化定理库的构建分为三个环节:
定理收集、定理标准化以及定理实例化。定理收集是原始教材或教辅资料中的定理、公式、公理等描述,
还有一部分是来源于标准答案中的解题过程;完成定理生成后,需要对定理进行规范化、标准化书写,这里
的规范化是指符合推理引擎的定理输入规范格式,方便推理引擎统一处理;定理实例化是将第二环节中的
规范化定理生成知识图谱,存储在规则库中。最后是图推理引擎的整体设计和实现,引擎在架构设计思想上
采用的分层结构,不同逻辑层之间相对独立,模块化程度高;在算法上的核心设计思想是图匹配以及置换
等价问题。最后为了提高引擎的处理问题的能力和稳定性,引入异常检测机制和冲突消解策略。综上可以将
本文中的研究工作及成果分为以下四个要点:

(1)构建了知识图谱中实体类和关系类,并用抽象类的设计思想实现了知识图谱构建的可扩展性,并
定义了抽象类的属性和结构,最终构建的知识图谱实体***********个,关系********条

(2)参与了实例化定理的搜集工作,制定了定理的规范化标准,并完成定理实例化,共
有实例化定理************条

(3)完成了整个推理引擎的设计和构建工作,包括其中的逻辑架构设计和实现,核心算法的设计和实现,
使用的推理方式设计和实现,每层的模块化构建,异常检测机制和冲突消解策略

(4)参与了类人解答过程的算法核心思想设计和部分构建工作。设计类人解答过程的分支回溯思想
,以及推理中的规则树构建。

\section{后续工作展望}
虽然本文完整了整个图推理引擎的设计和构建工作,并且也将推理引擎接入到系统中完成测试工作。
但是引擎在稳定性和兼容性上还存在不足,解题能力也有待提高。由于时间和精力有限,推理引擎中
存在的不足和疑难问题还待解决:

(1)匹配算法的精度问题。目前引擎中使用的匹配粒度是三元组,即三元组是匹配的最小单位。
而三元组采用是类型匹配,当将知识图谱拆成三元组的粒度时,会使图的部分信息丢失。后续工作中
准备采取关联三元组来解决这一问题,让三元组与三元组之间也产生关系依赖。

(2)知识爆炸问题。因为引擎是产生式系统的演变,因此会存在知识迭代,若匹配过程中
出现m个事实对应n个实例化规则模式,将出现组合情况,组合的知识继续参与迭代,迭代次数
过多将出现知识爆炸。后续工作准备在检测机制中加入虚假组合的检测,将无用组合通过检测机制消除掉,
减少知识迭代。

(3)知识冲突问题。推理引擎的第四逻辑层是知识更新,知识更新是完成将新知识插入到原始的知识库中。
当知识库存在于待插入的知识同类型的知识时,会出现知识冲突,如果不解决知识冲突问题,将会导致
知识插入失败,推理迭代出错,甚至会出现推理异常终止。后续工作时继续完善知识冲突消解策略,能够
处理和兼容到更多的知识冲突问题。

(4)推理效率问题。在引擎具备一定的稳定性和兼容性的前提下,需要考虑推理效率问题。
推理效率一般以时间效率和空间效率作为衡量标准。目前系统中影响推理效率主要来源于
三个方面:实例化定理的选取问题、分支组合数、引擎本身的算法。引擎中主要推理方式还是
正向推理,而正向推理是暴力搜索的方式,会出现很多无效匹配;分支组合数过多时,知识迭代
会引起知识爆炸,影响推理效率;引擎的算法需要优化核心的匹配算法。

\end{document}