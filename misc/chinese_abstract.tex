\documentclass{standalone}
% preamble: usepackage, etc.
\begin{document}
	
\begin{chineseabstract}
近些年,人工智能已经由传统的感知智能逐渐向认知智能阶段过渡,认知智能与自动推理成为很多专家学者
以及科研机构研究的重点。目前是处于认知智能发展的初期,面临着诸多挑战的同时也伴随着机遇。如何
将深度学习应用于逻辑推理,从而让机器具备思考和推理能力将是人工智能重大突破口。本文的研究内容是
基于图同构的初等数学推理引擎的设计和构建,推理引擎是整个自动推理系统最核心的部分。推理引擎的
系统设计理念是基于产生式系统,同样涉及到知识表示和实例化规则库构建两部分。具体研究内容如下:

(1)初等数学的知识表示

知识表示是类人解答系统求解问题的第一步,只有先将人类的语义转换成计算机的知识结构,才能进行后续的推理和
问题求解工作。知识表示的方法多种,由于本文研究课题对象是初等数学问题,分析和总结初等数学的知识具有概念
清晰严谨、定理明确以及公式化程度高等特点,最终选用知识图谱来表示数学中概念实体和它们之间的关系在实际编码中使用Java中的类结构来构建
初等数学的实体类和关系类,为了知识图谱构建的可扩展性和开闭性原则,选用抽象类的继承思想去扩展,即
所有的实体类都继承于抽象实体类,所有的关系类都继承于抽象关系类。

(2)构建实例化定理库

基于产生式系统的推理引擎,规则库是引擎重要的外部驱动和产生知识的依据。本文构建的规则库也是知识图谱的
方式表示和存储。实例化定理库的构建分为三个步骤:定理来源、定理标准化以及定理实例化。首先定理来源于
教材和教辅中的公式定理,以及标准答案的解答过程、解题技巧。对于搜索的定理是需要转换成推理引擎统一的
格式,这个过程叫定理标准化,方便推理引擎简单统一。最后将标准化定理生成知识图谱完成实例化。

(3)图同构的推理引擎的设计和构建
本文的推理引擎设计思想上可以从三个不同的角度去理解。首先是引擎中推理系统设计模式,是基于”先逆后正“
逻辑的产生式系统,并设计成逻辑推理与计算推理交互推理的混合推理;其次,从引擎逻辑结构考虑,引擎采用分层
结构,每个逻辑层之间保持相对独立性;最后,考虑引擎核心算法的设计,图匹配的算法思想在于是一种混合匹配算法,
知识图谱上通过类型匹配构建实体和关系的映射,然后再对字符串做模式匹配,生成符号轮换集。

本文完成了基于图同构的数学推理引擎的整个算法设计和所有模块的构建工作,完成非应用题的随机
批量测试和高考试卷测试,综合解题率71.2\%,平均求解时间不超过5分钟。


\chinesekeyword{认知智能,图匹配推理引擎,符号计算,类人解答系统,知识表示,实例化定理}
\end{chineseabstract}

\end{document}