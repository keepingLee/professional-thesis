\documentclass{standalone}
% preamble: usepackage, etc.
\begin{document}
	
\begin{chineseabstract}
近些年来,随着人工智能技术的飞速发展,许许多多专家学者开始将注意力集中在人工智能技术与教育的结合上。
同时,很多的科研机构和公司也致力于将人工智能技术赋能于教育。利用人工智能技术定义全新的教育模式,实现
计算机自动求解、自动辅导等功能。为学生提供快速、公平、系统并且量身定制的教育资源。这对提升教育质量,降低
教育成本和创新教育模式有着十分重要的意义\citing{wang1999sanwei}。让机器学会自动推理,这是目前认知智能
领域的一个重要目标,也是深度学习应用于逻辑推理的一个突破口。本论文的研究内容是仅在输入原始初等数学题目的
基础上,对题目进行自动类人求解,简称“高考机器人”。它主要包含如下几点:

1)图匹配推理引擎与符号计算推理的研究和构建

主要研究图匹配推理引擎与计算推理的逻辑架构与核心算法设计,针对出现问题给出相应的解决方案。从图匹配
推理引擎研究与构建,复杂逻辑与计算推理相融合和类人解答过程的输出三个方面进行。其中以图匹配推理引擎
的设计和实现为重点,讲述了三种不同的复杂逻辑推理组织方式,并采用“正逆结合”推理方式构建推理引擎。随
后研究了推理引擎与符号计算平台之间的联系,通过符号计算提供的计算服务为具体的问题的计算推理打下了支撑。
类人求解中,在推理的基础上,设计基于DFS的搜索算法,重构类人求解过程。

2)初等数学的知识表示

在自然语言处理(Nature Language Process,NLP)的支持下,本文主要研究对初等数学中的知识表示问题。
最终以三元组(实体-关系-实体)的形式表征文本的语义知识。

3)初等数学问题的实例化规则

这些规则主要分成两大类。第一类:初等数学的定理和公理,以及公式。第二类:是常见的解题技巧,公式变形,
以及解题策略。这些定理、公理、公式、策略以及解题技巧都演变成一个个规则实例。实例化规则通过统一的接口
全部生成实例化子图保存在neo4j图数据库中。这些规则库中每个规则子图都有唯一的标签,唯一的标签可以用来
获取相应的实例化规则子图。


\chinesekeyword{认知智能,图匹配推理引擎,符号计算,类人解答过程,知识表示,实例化规则}
\end{chineseabstract}

\end{document}